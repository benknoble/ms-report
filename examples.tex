\section{Learning by example}\label{S:examples}

I begin with an extended example comparing two proofs of different systems in
different verification environments (Section~\ref{S:ex_ext}). I then briefly
discuss a few other examples (Section~\ref{S:ex_notable}) and point towards a
plethora of other verified-programs research (Section~\ref{S:ex_reading}).

\subsection{Extended example: distributed hash-tables and regular
expressions}\label{S:ex_ext}

I have previously written two proofs of sufficient size and complexity that
their study will demonstrate both common proof techniques and challenges. I will
not provide all the details, in part because both were completed for academic
exercises and I wish to avoid publishing full solutions, and in part because
some details are not relevant for the discussion here (proof scripts and
programs available on request). It is important to note that in both cases,
there is no final executable program; rather, the proofs are written in a
relational style. This facilitates the proof at the cost of not having a program
to run. A final step would be to implement such programs and prove their
agreement with the relational definition.

The first proof is that a (somewhat idealized) distributed hash-table behaves
like a logically-centralized hash-table; this was modeled and proved in
Dafny~\cite{leino2010dafny} for the 2020 Systems Software Verification Summer
School~\cite{Kapritsos_2020}. I defer details on hash-tables to classic texts
like~\cite{CLRS}.

The second proof is that the theory of regular expressions entails the strong
pumping lemma; this was modeled and proved in Coq~\cite{Coq} for a course using
the \emph{Logical Foundations} text~\cite{Pierce:SF1}. Details on regular
expressions and the pumping lemma may be found
in~\cite{Lewis_1997,Morrisett_2012}\footnote{Many details on the derivatives of
regular expressions, also used in~\cite{Pierce:SF1,Morrisett_2012} but not
relevant here, can be found in~\cite{Might_Yacc,Might_desugar,Might_deriv}}.

I present the examples by following the chronology of their development. First,
I develop a model of the theory (Section~\ref{S:ex_theory}). Second, I model the
program to be verified (Section~\ref{S:ex_program}). Third, I write down the
formal specification (Section~\ref{S:ex_spec}). Finally, I prove by induction
that the models implement the specification (Section~\ref{S:ex_ind}).

\subsubsection{Model of the theory}\label{S:ex_theory}

\subsubsection{Model of the program}\label{S:ex_program}

\subsubsection{Formal specification}\label{S:ex_spec}

\subsubsection{Proof by induction}\label{S:ex_ind}

\subsection{Notable mentions}\label{S:ex_notable}

\subsection{Further reading}\label{S:ex_reading}

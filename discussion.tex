\section{Discussion}\label{S:discussion}

Having given an overview on verified programming and the tools used to build
verified programming languages, I turn now to the frontier of the science: what
challenges lie ahead? What open problems need answers?

We saw in Section~\ref{S:stack} that today's programming environment is more
expansive than a single program on a single machine.
IronClad~\cite{hawblitzel2014ironclad} and
IronFleet~\cite{hawblitzel2015ironfleet} address full-stack distributed
verification, but I have not found any research on build-pipeline or deployment
verification, nor on containerization, orchestration, and the like. On the one
hand, we might hope that deployment steps are simple enough that the
cost-benefit ratio of formal verification indicates the effort is not worth the
confidence. Yet the mishaps that occur on mis-deployment might indicate
otherwise, and the complexity of rollouts for some teams indicate a need for
increased confidence. With respect to a build-pipeline, verified compilers like
CompCert~\cite{Kastner-LBSSF-2017} yield verified compilation stages, but there
are often other steps before and after compilation that are left unverified
(they may, of course, be tested). There may be room to grow in this direction as
well. Lastly, verified containerization technology would improve on today's
state of systems based on Docker~\cite{merkel2014docker} and
Kubernetes~\cite{k8s} by being able to guarantee the security and isolation
promises of containerization, just like RustBelt~\cite{Jung_2018a} is now able
to guarantee the safety promises of Rust.

In the same section, we peeked into the deeper layers, like the executable
machine code, the \gls{os}, and the hardware. Translation
validation~\cite{Pnueli_1998} and verified compilers give confidence in the
machine code, but CompCert still lacks a fully verified assembler and linker. A
number of projects are researching verified \gls{os} and kernels (see
Section~\ref{S:background} for a list) and discovering their own unique
challenges in design and proof-management~\cite{Klein_AEMSKH_14}. Hardware
presents verification challenges different from those of software; some current
research is examining security properties and symbolic execution \`{a} la
Rosette for hardware~\cite{zhang2018end}.

In Section~\ref{S:t_interaction} we saw 3 types of verification tools
(interactive, auto-active, and automatic). In our examples from
Section~\ref{S:ex_ext}, we saw a comparison between one interactive proof and
one auto-active proof, where both had benefits. Tools like Rosette and other
push-button approaches bring yet more automation, at certain costs (\eg,
finitization). Developing better automation tools for the interactive provers
(\eg, Mtac~\cite{Kaiser_2018}, Iris proof mode~\cite{Krebbers_2017b}) will bring
them closer to push-button tools, making them more accessible to an average
developer. On the other hand, developing tools that provide more robust insights
to the automatic provers (\eg,
profiling~\cite{Bornholt_2018,Porncharoenwase_2020}) can help the programmer
solve automation-related problems, especially exponential path explosion, scale,
and timeout-demons. In a related vein, we saw the use of \texttt{\{:opaque\}} to
combat automation issues related to quantifiers, and we also saw that Coq
struggles to rewrite under the existential quantifier, demonstrating needs for
improvements on these facets of the system.

Iris and Section~\ref{S:t_logic} demonstrated that proof logics (\eg, extensions
to Hoare-logic) are still under active development: while systems often need to
build their own proof logics, checking that they are sound is a difficult
process. Further, explaining and using them requires deep background. What new
proof logics will we need to keep up with the state of programming technology?
Iris was developed in part specifically for Rust; it also managed to position
itself as a foundation for several older, more complex proof logics. As our
language technology evolves, we will need increasingly sophisticated logics
while we also hope to keep them accessible to the programmer writing proofs.
Another possible direction here is to enable more rapid proof development:
when developing a language model for a proof, like Rust, the programmer has to
tie the syntax, semantics, and proof logic together. Facilitating these steps
would help automate early development and allow the programmer to focus
more on the proofs of interest. It would also enable a tighter feedback loop
between model changes and proof updates.

Section~\ref{S:t_pl} mentions several programming languages or paradigms along
with their accompanying verification tools. Bringing verification to more
languages and more languages to verification will remain an open problem for as
long as we continue to develop new programming languages. One technique for
doing verifying new languages is to lift a non-verified language into an
already-verified language~\cite{leino2008specification}. A particular challenge
when modelling new languages is to verify that the model is accurate to the
language specification and implementation; CompCert provides a reference
interpreter~\cite{Kastner-LBSSF-2017} for that purpose, though doing so is
challenging~\cite[Chs. \emph{Simple Imperative Programs}, \emph{An Evaluation
Function for Imp}]{Pierce:SF1}. In a similar vein, verifying compilation or
interpretation for more languages would bring increased confidence to systems
using those languages. While avionics is one example of safety-critical
applications that relies on formal techniques, applications like financial
software may not use them to their full extent, in part due to a lack of
appropriate technology.

When we looked at the extended examples in Section~\ref{S:ex_ext}, we saw early
on that some programs and proofs produce executable programs, while others are
only models that cannot actually be run. Proof by refinement can later tie an
executable to the model, but refinement is often a meta-technique: there is no
built-in support for the technique, so the programmer must build out both the
technique and the argument that it is applicable. Similarly, the inductive
state-machine model is better supported by some languages (\eg, Coq's inductive
types) than others (\eg, Dafny's relation predicates, which are not tied
together by any meaning other than what the programmer ascribes), which also
impacts the statement of the specification. We also saw that graph and
bit-vector proofs are unwieldy in some languages~\cite{Morrisett_2012}.

We do see in the extended examples one use of parametric or abstract modules;
Iris takes a similar approach, allowing the programmer to instantiate the module
with specifics of the problem. The concept comes from ML's module system;
specifically, it is analogous to functors in ML and OCaml, which are
module-level functions from modules to modules. Developing further abstraction
mechanisms might prove fruitful for some of the problems of proof development
and automation.

Other research demonstrated open problems in various related fields. For
example, RockSalt~\cite{Morrisett_2012} shows a lack of strong tooling for
assembly-language semantics and related verification. Their DSL development
takes promising steps towards improving the state of assembly verification. It
also exposed questions of scale, portability, and testing (see
Section~\ref{S:ex_notable}). Rosette identified challenges related to symbolic
path explosion~\cite{Bornholt_2018,Porncharoenwase_2020}. CompCert examined the
problems of verifying linkers and assemblers in addition to questioning the
extraction mechanism. Bootstrapping verified compilers is a studied but
non-trivial technique~\cite{Konat_2016,Myreen_2021}, and the goal for CompCert
would to extract to verifiable C. Jitterbug~\cite{258848} raises questions about
integrating proof-developments into live software like the Linux kernel, where a
clean-slate approach is impossible. They also reinforce the challenge of
symbolic execution. One facet of the Jitterbug development involved reasoning
across different verifiers (in particular, interactive and automatic), an area
which has seen little research to my knowledge. Serval examined the challenge of
separating implementation and verification while maintaining the necessary
connections~\cite{Nelson_2019}. Bedrock, in a vein similar to Serval and
CompCert, took one of the first looks at cross-language
verification~\cite{Wang_2014}.

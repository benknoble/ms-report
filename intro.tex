\section{Introduction}

First, I introduce the ideas of formal methods and formal verification, in
particular their connection to programming. Next, I briefly dissect the various
layers of abstraction that programmers operate in and the corresponding
layers of verification. Finally, I outline the direction of the rest of the
paper.

\subsection{Background}

Formal methods are mathematical techniques for describing and verifying system
properties~\cite{Wing_90}. They have been successfully applied to the practice
of programming and computation since the birth of computing:
\citeauthor{Turing_1937}'s famous 1937 discussion of the {\haltprob} is an
application of mathematical reasoning to a generalized system of
computation~\cite{Turing_1937}. The use of formal methods continued throughout
the modern computing era, with \citeauthor{McCarthy_67} formalizing the first
proof of a verified compiler in 1967~\cite{McCarthy_67}. The proof was later
mechanically verified to be correct in 1972~\cite{Milner_72}. The computing
giant and prolific writer \citeauthor{EWD:EWD1036} argued in 1988 that the very
nature of programming depended on notions of symbolic manipulation, formal
semantics, and the task of giving a ``formal proof'' that the proposed program
``meets the equally formal functional specification''---and that Computer
Science education at the time generally omitted much of the relevant background
and material necessary for accomplishing this task~\cite{EWD:EWD1036}.

Formal verification, \citeauthor{EWD:EWD1036}'s act of proving that the program
implements its specification, has not died as he predicted through a ``foggy
crystal ball.'' Instead, they have become increasingly powerful, accessible, and
vital, as I aim to elucidate here. Some computer scientists hesitate to make
claims about the termination of a particular program~\cite{Cook_2011}, perhaps
intimidated by \citeauthor{Turing_1937}'s Halting Problem arguments; yet, proofs
of termination, correctness, non-interference, and more are increasingly
necessary and useful to reason about programs. For example, such proofs are now
used to reason about OS
kernels~\cite{Klein_EHACDEEKNSTW_09,Klein_AEHCDEEKNSTW_10,Klein_AEMSKH_14,Sewell_KH_16,Narayanan_2019,Narayan_2020,Nelson_2017},
safety-critical systems such as avionics~\cite[\S 1]{Leroy-Compcert-CACM},
in-kernel compilers~\cite{186144,258848}, file-systems~\cite{Zou_2019}, and
concurrent systems~\cite{222565,222621}. These are just a few modern examples in
systems-contexts; verification has made its way into everyday programming, with
applications to, \eg, web-scraping, spatial programming, and superoptimization
for bitvector programs~\cite[\S 4]{Torlak_2013}.

In parallel to the growing efforts of formal verification, rising complexity
presents ongoing challenges to the development and correctness of software. This
is unfortunately coupled with the increasing power of verification tools---the
latter is necessary to make verification of the formal amenable. Consequently,
the verification tools also exhibit increasing complexity, with some being
essentially expert systems\footnote{For a glance at the complexity of some
systems, see~\cite{Jung_2015,Jung_2016,Krebbers_2017a,Jung_2018b} and
\figurename~\ref{F:iris_complex}. In fairness, as we'll see, this complexity
arises from distilling a family of complexities to a simpler base, and then
using that to prove Rust's safety~\cite{Jung_2018a}.}. Other systems combine
automation with domain knowledge to create a narrower, more accessible tool. We
will see some examples of this complexity and automation in
Section~\ref{S:examples}.

\subsection{The Verified-Programs Stack}

To examine the technology behind verified programs, it will be helpful to
understand the layers of abstraction in today's programming environments and the
corresponding verification layers. I present a brief overview of these layers,
with notes on what I will cover in the remainder of the paper and what is out of
scope; where possible, references are provided for further reading.

The general idea is that, in order to fully trust a program implements its
specification, we should like a proof at the source level and proofs that each
of the intervening levels between source and final execution preserve the
validity of the proof; in other words, that the final result can still be said
formally to implement the specification without the work of repeating the proof
at each stage.

The model presented in \figurename~\ref{F:abstraction} generally ignores
distributed systems and newer environments like the cloud; here I am primarily
concerned with a program running on a single machine, though it is possible to
generalize the model. I also ignore containerization, orchestration, and build
and deployment pipelines insofar as they are represented as user-space programs
or compositions thereof. Build pipelines naturally include compilers, which are
shown in the model despite being user-space programs. Much of the technology we
will see in this paper has been or can be extended to these domains.

Also out of scope is full-stack verification in the vein of
IronClad~\cite{hawblitzel2014ironclad} and
IronFleet~\cite{hawblitzel2015ironfleet}. Here I am considering the individual
components, especially the top layers.

\begin{figure}[hb]
    \centering
    \begin{tikzpicture}[stack/.style={rectangle split, rectangle split parts=5, draw, anchor=center}]
        \node[stack](prog){%
            \nodepart{one}Program source
            \nodepart{two}Compiler or Interpreter
            \nodepart{three}Executable Machine Code
            \nodepart{four}OS and Kernel
            \nodepart{five}Hardware
        };
        \node[stack,right=of prog](verification){%
            \nodepart{one}Program verification
            \nodepart{two}Verified compilers or interpreters
            \nodepart{three}Translation Validation
            \nodepart{four}Verified OS and Kernel
            \nodepart{five}Verified Hardware
        };
        \draw[<->] (prog.one east) -- (verification.one west);
        \draw[<->] (prog.two east) -- (verification.two west);
        \draw[<->] (prog.three east) -- (verification.three west);
        \draw[<->] (prog.four east) -- (verification.four west);
        \draw[<->] (prog.five east) -- (verification.five west);
    \end{tikzpicture}
    \caption{A simplified programming environment and execution stack, with corresponding verification technology}\label{F:abstraction}
\end{figure}

At the very top of the stack, we have the program source. Whether high-level or
low-level, this is the definition of what we want the computer to do.
Verification requires a formal specification of that behavior and a proof that
the source-as-written, coupled with the language's formal semantics, implement
that behavior. We will study almost entirely tools and techniques for
accomplishing this task. Some key components include
\begin{inlist}
\item the kind of verifier (\eg, push-button or interactive);
\item the logic of the verifier (\eg, \gls{smt} or higher-order constructive
    propositional logic via \gls{coc}); and
\item the logical system of the proof (\eg, separation logic).
\end{inlist}

In order to execute the program, we have the remaining layers. In order:
\begin{enumerate}
    \item\label{i:stack_translator} A program translator (compiler or
        interpreter)\footnote{These are almost eschewed in the case of the
        assembly programmer; yet, there is still the assembler or linker to
        contend with.}
    \item\label{i:stack_asm} The produced executable machine code (\eg, ELF
        binaries)
    \item\label{i:stack_OS} The OS and kernel responsible for managing execution
        of the machine code
    \item\label{i:stack_hardware} The underlying hardware that carries out the
        instructions
\end{enumerate}

Items~\ref{i:stack_OS} and~\ref{i:stack_hardware} are out of scope for this
paper. I refer the interested reader to research on verified OS and
kernels~\cite{Klein_EHACDEEKNSTW_09,Klein_AEHCDEEKNSTW_10,Klein_AEMSKH_14,Sewell_KH_16,Narayanan_2019,Narayan_2020,Nelson_2017}
and on verifying
processors~\cite{sturton-memocode13,Sturton_2013,Bradfield_2016,zhang2017identifying,zhang2018recursive,zhang2018end}.

Item~\ref{i:stack_asm} is particularly interesting and has generated much work
on translation validation~\cite{Pnueli_1998}. This kind of verification proceeds
by validating each run of a compiler, comparing the executable code to the
source and guaranteeing that a \emph{particular} set of executable code
implements a \emph{particular} set of program source code. I will have only
slightly more to say on this subject when we get to verified compilers; I point
the interested reader to works such
as~\cite{Sewell:phd,Sewell_KH_16,Sewell_2013,Necula_2000}.

This leaves item~\ref{i:stack_translator}, the program translator. In
safety-critical systems, it is vital that there be no miscompilation, \ie, bugs
introduced by compilation. Even outside of such restrictive domains, it is
important to trust that the compiler faithfully compiles source code; otherwise,
our systems are all subject to vicious attack~\cite{Thompson_1984}. Thus a
verified compiler\footnote{with verified interpreters a combination of an
on-the-fly compiler and a user-space program, though they have their own unique
challenges} is an important component in the software-development tool-chain.
Verifying a compiler amounts to both a proof about a particular program---the
compiler itself---\emph{and} a proof about a \emph{class} of programs---those
accepted and generated by the compiler. One hopes that a verified compiler
preserves proofs about the source; that is, if a program \(S\) is proven to have
some property \(P\) (written \(S \models P\) after~\cite{Leroy-Compcert-CACM}),
compiling with a verified compiler \(C\) should imply that \(C(S) \models P\).
Thus we have a statement about an entire class of programs, and it is this type
of statement that allows the composition of proven layers to itself be proven.

\subsection{Structure}

In Section~\ref{S:categories} I propose to classify the tools used to verify
programs. In Section~\ref{S:examples} I look at a few examples of verified
programs and identify a few prominent strategies and challenges. In
Section~\ref{S:discussion} I discuss the frontier of the formal-methods and
verified-programs research, attempting to answer the question ``What are the
next big challenges to tackle for program verification''. Finally, I conclude
with Section~\ref{S:conclusion}.

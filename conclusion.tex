\section{Conclusion}\label{S:conclusion}

In this report, I have examined the state of research and development of
formally verified programs and the technology behind them. I started with a
historical background on formal methods before jumping to a modern context
(Section~\ref{S:background}). I explored a fragment of today's programming
environment with respect to formal verification, specifically, user-programs and
compilers (Section~\ref{S:stack}). I proposed three (3) axes along which to
categorize verified programming tools:
\begin{inlist}
\item interactivity (Section~\ref{S:t_interaction}),
\item proof logic (Section~\ref{S:t_logic}), and
\item programming language (Section~\ref{S:t_pl}).
\end{inlist}
In particular, I examined the differences between interactive, auto-active, and
automatic verifiers. I also discussed the prominent proof logic known as Hoare
logic, along with a few variations like separation and concurrent separation
logic. I then presented two (2) extended examples to demonstrate common proof
architectures and challenges (Section~\ref{S:ex_ext}). I focused specifically on
proofs by induction and reflection, on identifying the correct specification, on
the contrasts between the proof environments, and on the challenges of
automation. I then provided an overview of a large number of research projects,
the problems they tackle, and the challenges they expose
(Section~\ref{S:ex_notable}--\ref{S:ex_reading}). Lastly, I summarized the grand
challenges of formal verification today: namely, challenges of automation,
deployment, trusted software and hardware, program and proof logic, development
ease and proof management, symbolic path explosion, and cross-language
verification (Section~\ref{S:discussion}).

I thus conclude the report on two notes: first, the last decade has seen an
explosion of formal-verification research, leading to advanced development
techniques and automation. We are growing closer to \citeauthor{EWD:EWD1036}'s
dream of commonplace formal verification even as new waves of abstraction and
generalization are invented. Second, formal verification is a growing body of
science---and this means it has a growing body of open problems. The last decade
has also seen an explosion of computing at scale, and one of the challenges of
today's verification efforts will be to keep up, be it through novel automation
and scaling techniques, more robust proof technology, or verified deployments.
The same explosion has happened in the programming language world, so
verification efforts will need to keep pace to bring verification to more
languages and more languages to verification.

The future of verification will also depend on good judgement: as~\cite{258848}
reminds us, knowing what \emph{not} to verify is just as important as knowing
what to verify. In some cases, informal arguments, unit testing, or other
confidence-building disciplines are enough effort for value. In others, the
benefits of formal verification outweigh the costs, either because the
verification is automatic or because it is necessary. As formal verification
becomes yet another tool in the programmer's toolbox~\cite{Brooks_1996}, the
programmer will need to decide justly whether a problem needs the hammer of a
proof or the light touch of another tool.

% Preamble {{{

% newcommands, DeclareMathOperators, &c. {{{
\theoremstyle{plain} % default
\newtheorem{thm}{Theorem}[section]
\newtheorem{lem}[thm]{Lemma}
\newtheorem{prop}[thm]{Property}
\newtheorem*{cor}{Corollary}

\theoremstyle{definition}
\newtheorem{defn}{Definition}[section]
\newtheorem{example}{Example}[section]
\newtheorem{exercise}{Exercise}[section]
\newtheorem*{prob}{Problem}

\theoremstyle{remark}
\newtheorem*{rem}{Remark}
\newtheorem*{note}{Note}
\newtheorem{case}{Case}

\newcommand{\yields}{\Rightarrow}
\newcommand{\Yields}{\Rightarrow^{\star}}
\newcommand{\union}{\cup}
\newcommand{\intersect}{\cap}
\newcommand{\Union}{\bigcup}
\newcommand{\Intersect}{\bigcap}
\newcommand{\compose}{\mathrel{\circ}}
\newcommand{\comp}[1]{{\overline{#1}}}
\newcommand{\DownTo}{\mathrel{\Downarrow}}
\newcommand{\To}{\mathrel{\Rightarrow}}
\newcommand{\TO}{\mathrel{\Rrightarrow}}
\newcommand{\definedas}{\triangleq}

\newcommand{\bb}[1]{\mathbb{#1}}
\newcommand{\N}{\bb{N}}
\newcommand{\Z}{\bb{Z}}
\newcommand{\Q}{\bb{Q}}
\newcommand{\R}{\bb{R}}
\newcommand{\I}{\bb{I}}

\newcommand{\mc}[1]{\mathcal{#1}}
\newcommand{\Powerset}[1]{\mc{P}\left(#1\right)}

\newcommand{\set}[1]{\left\{#1\right\}}
\newcommand{\setbuild}[2]{\left\{ #1 : #2\right\}}

\newcommand{\var}[1]{\mathit{#1}}
\newcommand{\func}[1]{\mathit{#1}}

\DeclareMathOperator{\false}{false}
\DeclareMathOperator{\true}{true}

\DeclarePairedDelimiter{\abs}{\lvert}{\rvert}
\DeclarePairedDelimiter{\norm}{\lVert}{\rVert}
\DeclarePairedDelimiter{\ceil}{\lceil}{\rceil}
\DeclarePairedDelimiter{\floor}{\lfloor}{\rfloor}

\DeclareMathOperator{\lcm}{lcm}

\newcommand{\idem}{\emph{idem.}}
\newcommand{\vs}{{\emph{v{.}}}}
\newcommand{\ie}{\emph{i.e.}}
\newcommand{\eg}{\emph{e.g.}}
\newcommand{\etc}{\emph{etc.}}
\newcommand{\etal}{\emph{et al.}}

\newcommand{\haltprob}{\emph{Entscheidungsproblem}}
% newcommands, DeclareMathOperators, &c. }}}

% Preamble }}}
